\section{Introduction}
The rapid advancements in semiconductor technology have continually driven the need for more efficient power delivery mechanisms in integrated circuits (ICs). As transistor dimensions shrink, traditional power delivery methods face significant challenges, including increased power density and reduced routing resources. Advanced transistor architectures like FinFETs and Gate-All-Around (GAA) offer substantial improvements in power efficiency and scalability but introduce new complexities in power delivery due to their unique structural characteristics.

\subsection{Challenges in Power Delivery for Advanced Nodes}
As ICs advance to smaller process nodes, the power delivery network (PDN) must evolve to meet the increasing demands for power efficiency and performance. The traditional frontside power delivery networks are often insufficient due to their longer and narrower interconnects, which result in higher resistive losses and voltage drops. The need for innovative solutions has led to the exploration of Backside Power Delivery (BPD) networks, which propose relocating the power grid to the backside of the chip, thereby offering potential benefits in terms of reduced power loss and enhanced efficiency.

\subsection{Backside Power Delivery (BPD)}
Backside Power Delivery (BPD) has emerged as a promising solution to address the limitations of traditional PDNs. By providing a dedicated power network on the backside of the chip, BPD enables shorter, wider interconnects, thus minimizing power loss and improving voltage stability. This innovative approach has shown potential in reducing power delivery inefficiencies, particularly in advanced FinFET and GAA architectures.

\subsection{Industry Advancements and Research}
Recent studies and industry advancements highlight the viability and benefits of BPD networks. For instance, the work by H.J. Chia et al. (2023) on ultra-high density low-temperature wafer-on-wafer SoIC bonding demonstrates significant improvements in uniform resistance distribution and reliability, which are critical for effective power delivery in advanced nodes. Similarly, the panel discussion on EDA challenges at advanced technology nodes underscores the importance of addressing system design, verification, thermal management, and reliability for successful implementation of BPD networks.

The research by C. Durfee et al. (2023) on damage mitigation in stacked nanosheet GAA transistors further emphasizes the need for robust thermal management and reliability strategies in advanced semiconductor nodes. Additionally, A. Nathan and colleagues (2023) provide a comprehensive overview of advancements in transistor technology, including tools for performance and reliability characterization, which are essential for optimizing BPD networks.

\subsection{Objectives of the Thesis}
This thesis aims to investigate the design, optimization, and implementation of BPD networks for FinFET and GAA architectures at advanced process nodes. The primary objectives include:
\begin{itemize}
    \item Comprehensive Analysis of BPD for Advanced Nodes: Investigate the design, optimization, and implementation challenges of BPD for FinFET and GAA technologies.
    \item Performance and Scalability Evaluation: Quantify the impact of BPD on performance, power efficiency, and scalability of advanced ICs.
    \item Design and Optimization of BPD Networks: Develop novel design and optimization methodologies to maximize power delivery efficiency and minimize area overhead.
    \item Thermal and Reliability Analysis: Analyze the thermal and reliability implications of BPD for advanced ICs and develop mitigation strategies.
\end{itemize}
By addressing these objectives, this research seeks to provide a comprehensive understanding of the potential of BPD networks to revolutionize power delivery in next-generation ICs.
